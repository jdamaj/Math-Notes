
\subsection{More Fundamental Groups}

\begin{itemize}
    \item We can write $S^n$ as the union of two open sets which are simply connected (homeomorphic to $\bR^n$) and where intersection is path connected connected so $\pi_1(S^n, *)=0$ 
    \item Similarly the figure eight space can be written as two open subsets which acn be retracted to the circle so its fundamental group is generated by loops around the two circles. 
\end{itemize}

\noindent
Suppose $X,Y$ path connected, then $X \times Y$ is path connected. 

\begin{theorem}
    Pick base points $*_X, *_Y$ for $X$ and $Y$ use $(*_X, *_Y)$ as a basepoint for $X \times Y$, then $\pi_1(X \times Y, (*_X, *_Y)) = \pi_1(X, *_X) \times \pi_1(Y, *_Y)$ 
\end{theorem}

\begin{pf}
    If $h$ is a loop in $X \times Y$ at $(*_X, *_Y)$, $h: I \to I \times Y$, then $h(t) = (f(t), g(t))$ where $f$ is a loop in $X$ at $*_X$ and $g$ is a loop in $Y$ at $*_Y$ so we have a map $\pi_1(X \times Y, (x_*, y_*)) \to \pi_1(X, *_X) \times \pi_1(Y, *_Y)$.If $f$ is a loop in $X$, then set $h(t) = (f(t), *_Y)$. From this we see that map $\pi_1(X \times Y, (x_*, y_*)) \stackrel{\phi}{\to} \pi_1(X, *_X) \times \pi_1(Y, *_Y)$ is surjective. To show that $\phi$ is injective, show that its kernel is trivial. Given $h$, with $\phi([h]) = (0,0) \in \pi_1(X) \times \pi_1(Y)$, $h=(f,g)$ so this implies that $[f]=0$, $g[0]$ so $f \sim *_X$ and $g \sim *_Y$ so $h \sim (*_X, *_Y)$ so $[h]=0$. 
\end{pf}

\subsection{Group Actions}

Let $G$ be a group and let $\alpha$ bea an action of $G$ on a topological space $X$, $\alpha \to \text{Homeo}(X)$, For $x \in X$, its orbit is $\{ \alpha_g(x) \, : \, g \in G\}$. The orbits form a partition of $X$. We let $X/\alpha$ be the set of orbits. Can give $X/\alpha$ the quotient topology. \\
$X/\alpha$ can be bad. Given $X = \{z \in \bC \, : \, |z| =1\}$, let $r_0$ be an irrational number. Let $\alpha$ be the action of $\bZ$ on $X$ by $\alpha_n(z) = e^{2 \pi i r_0}z$, all orbits are dense. 

\begin{proposition}
    $p: X \to X/\alpha$ is an open map. 
\end{proposition}

\begin{proof}
    For $x \in X$, $p^{-1}(p(X)) = $ the orbit of $X$. If $\cO \subseteq X$ is open, $p(\cO)$ open? $p^{-1}(p(\cO))$ open = union of the orbits of the points in $\cO = \bigcup_{g \in G} \alpha_g(\cO)$ open. 
\end{proof}

\begin{definition}
    $\alpha$ is free in $G_x = \{e_G\}$ for all $x$, where $G_x = \{g \in G \, : \, \alpha_g(x)=x\}$, then $g \mapsto \alpha_g(x)$ is a bijection from $G$ onto the orbit for all $x$. 
\end{definition}

\begin{itemize}
    \item the irrational rotation action is free. 
\end{itemize}

\begin{definition}
    An action $\alpha$ of $g$ on $X$ is free and proper if for each $x$ there is an open set $\cO$, $x \in \cO$ such that the sets $\alpha_g(\cO) = \{\alpha_g(y) \, : \, y \in \cO\}$ are all disjoint ie $X \stackrel{p}{\to} X/\alpha$ is a cover, for $\cO$, $p^{-1}(\cO) = \oplus_{g \in G} \alpha_g(\cO)$ 
\end{definition}

\begin{example}
    $X = S^n$, $G = Z_2 = \{0,1\}$, $\alpha_1(v)=-v$ $S^2 /\alpha = P^2$ $S^2$ simply connected so $\pi_1(P^2) = Z_2$. 
\end{example}