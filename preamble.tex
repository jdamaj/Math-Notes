% basics
\usepackage[utf8]{inputenc}
\usepackage[T1]{fontenc}
\usepackage{textcomp}
\usepackage{pgffor}
\usepackage{xcolor}
\usepackage{amsmath}
\usepackage{enumitem}
\usepackage{amssymb}
\usepackage{tikz}
\usetikzlibrary{positioning}
\usetikzlibrary{shapes.geometric}
\usetikzlibrary{decorations.pathmorphing}
\usetikzlibrary{decorations.markings}
\usetikzlibrary{arrows.meta,bending}
\usetikzlibrary{decorations.pathreplacing,calligraphy}
\usepackage{graphicx}
\usepackage{float}
\usepackage{tikz-cd}
\usepackage{bookmark}
\usepackage{hyperref}
\usepackage{stmaryrd}
\usepackage{mathabx} 
\usepackage{mathtools} 



%%%%%%%%%%%%%%%%%%%%%%%%%%%%% Doc Set UP

\usepackage{fancyhdr}
\renewcommand{\headrulewidth}{0.5pt}
\fancypagestyle{contentpage}{%
	\lhead{\rightmark}
	\cfoot{\thepage}
}

\definecolor{wikipediadarkblue}{rgb}{0.023, 0.270, 0.676}
\hypersetup{
	colorlinks,
	citecolor=black,
	filecolor=black,
	linkcolor=wikipediadarkblue,
	urlcolor=wikipediadarkblue,
}


%%%%%%%%%%%%%%%%%%%%%%%%%%


%%%%%%%%%%%%%%%%%%%%%%%%%% Theorems


\usepackage{amsthm}
\usepackage{thmtools}
\usepackage[framemethod=TikZ]{mdframed}
\mdfsetup{skipabove=1em,skipbelow=0em, innertopmargin=5pt, innerbottommargin=6pt}


\declaretheoremstyle[
  mdframed = {
      linewidth = 0.5pt;
  },
  headfont=\normalfont\bfseries,
  bodyfont=\normalfont,
]{mystyle}

\declaretheoremstyle[
  mdframed = {
      linewidth = 0.5pt,
      rightline = false,
      topline = false,
      bottomline = false,
  },
  headfont=\normalfont\bfseries,
  bodyfont=\normalfont,
]{pfstyle}

\declaretheoremstyle[
  headfont=\normalfont\bfseries,
  bodyfont=\normalfont,
]{reg}


\declaretheorem[style=mystyle, name=Theorem,within=section]{thm}
\declaretheorem[style=mystyle, name = Theorem, sibling = thm]{theorem}
\declaretheorem[style=mystyle, name = Definition, sibling = thm]{definition}
\declaretheorem[style=mystyle, name = Proposition, sibling = thm]{proposition}
\declaretheorem[style=mystyle, name = Lemma, sibling = thm]{lemma} 
\declaretheorem[style=mystyle, name = Corollary, sibling = thm]{corollary}
\declaretheorem[style=mystyle, name = Axiom, sibling = thm]{axiom}

\declaretheorem[style = pfstyle, name = Proof, numbered = no]{pf}

\declaretheorem[style =reg, name = Remark, sibling = thm]{remark}
\declaretheorem[style = reg, name = Notation, sibling = thm]{notation}
\declaretheorem[style = reg, name = Example, sibling = thm]{example}
\declaretheorem[style = reg, name = Lemma, sibling = thm]{nblemma} 
\declaretheorem[style = reg, name = Exercise, sibling = thm]{exercise}

\declaretheorem[style = reg, name = Fact, sibling =thm]{fact}

%%%%%%%%%%%%%%%%%%%%%%%%%%%


%%%%%%%%%%%%%%%%%%%%%%%%%%%%% Comands

\newcommand{\bZ}{\mathbb{Z}}
\newcommand{\bR}{\mathbb{R}}
\newcommand{\bQ}{\mathbb{Q}}
\newcommand{\bF}{\mathbb{F}}
\newcommand{\bC}{\mathbb{C}}
\newcommand{\bN}{\mathbb{N}}
\newcommand{\cP}{\mathcal{P}}
\newcommand{\ex}{\exists}
\newcommand{\fa}{\forall}
\newcommand{\from}{\leftarrow}
\newcommand{\vspan}{\text{span}}
\newcommand{\Hom}{\text{Hom}}
\newcommand{\Endo}{\text{End}}
\newcommand{\img}{\text{im}}
\newcommand{\ifff}{\leftrightarrow}
\newcommand{\vp}{\varphi}
\newcommand{\ve}{\varepsilon}

\newcommand{\rotaterelation}[1]{\rotatebox[origin=c]{-90}{$\mathstrut#1$}}
\newcommand*\circled[1]{\tikz[baseline=(char.base)]{\node[shape=circle,draw,inner sep=2pt] (char) {#1};}}

\newcommand{\cU}{\mathcal{U}}
\newcommand{\cV}{\mathcal{V}}
\newcommand{\cT}{\mathcal{T}}
\newcommand{\cL}{\mathcal{L}}
\newcommand{\cC}{\mathcal{C}}
\newcommand{\cF}{\mathcal{F}}
\newcommand{\cR}{\mathcal{R}}
\newcommand{\cM}{\mathcal{M}}
\newcommand{\cN}{\mathcal{N}} 
\newcommand{\cB}{\mathcal{B}}
\newcommand{\cS}{\mathcal{S}}
\newcommand{\cI}{\mathcal{I}}
\newcommand{\cO}{\mathcal{O}}
\newcommand{\cA}{\mathcal{A}}
\newcommand{\cD}{\mathcal{D}}
\newcommand{\cG}{\mathcal{G}}

\newcommand{\vn}{\varnothing} 
\newcommand{\card}{\text{card}}
\newcommand{\dom}{\text{dom }}
\newcommand{\ran}{\text{ran }} 

\newcommand{\dia}{\text{Diag}} 
\newcommand{\diae}{\text{Diag}_{\text{el}}} 
\newcommand{\tr}{\text{tr}}

%%%%%%%%%%%%%%%%%%%%%%%%%%%%%%%%
