
\subsection{Polynomials}

Recall: 
\begin{enumerate}
    \item Polynomials over a field have a Euclidean division algorithm \\ 
    For $f,g$, $f = gq+r$, $\deg (r) \le \deg (g)$ ($g \neq 0$), $\deg(0) = - \infty$ 
    \item $k[x]$ has a unique factorization
\end{enumerate}

\noindent
Primes of $\bZ \ifff $ irreducible polynomials 

\begin{example}
    Sieve of Eratosthenes: \\
    on $\bZ$: $\not1$, $\circled{2}, circled{3}, \not4, circled{5}, \not6, circled{7}, \not8, \not9, \not{10} \cdots$ \\
    on $\bF_2[x]$: $\not1$, $\circled{x}, \cricled{x+1}$, $\not{x^2}$, $\not{x^2+1}$, $\not{x^2+x}$, $\circled{x^2+x+1}$, $x^3 \cdots$ \\
    only need to write polynomials such that constant term nonzero, sum of coefficient is odd. \\
    $\circled{x^2+x+1}$, $\circled{x^3+x+1}$, $\circled{x^3+x^2+1}$, $\circled{x^4+x+1}$, $\not{x^4+x^2+1}$, $\circled{x^4+x^3+1}$, $\not{x^4+x^3+x^2+x+1}$ 
\end{example}

\noindent
Recall: If $f(x)$ has a root $x-a$. $f(x) = (x-a)g(x)$ since $f(x) = (x-a)g(x) + r$, $\deg f \le 0$.\\
 If $R$ is an integral domain, polynomial in $R[x]$ has $\le \deg f$ roots. \\
 This is false in general: $R = \bZ/8\bZ$ $f(x^2-1)$ has 4 roots $x=1,3,5,7 \mod 8$ \\

 \begin{corollary}
    $(\bZ/p\bZ)^*$ cyclic, prime $p$ has primitive roots. 
 \end{corollary}

 \begin{pf}
    We show that a finite subgroup of $F^*$ ( for a field of $F$) is cyclic. $G$ has $\le n$ elements, with $g^n=1$ (any $n \ge 1$) since polynomial $x^n-1$ has $\le n$ roots so by the structure theorem for abelian groups $G = \bZ/n_1\bZ \times \bZ/2\bZ \times \cdots$ $n_2|n_1, n_3|n_2, \ldots$. If $n_2 >1$, $G$ has $\ge n_2^2$ elements of order $n_2$ so $n_2 = n_3 = \cdots = 1$ 
 \end{pf}   

 \begin{example}
    $F = \bZ/7\bZ$ $F^* = 1, 2, 3, 4, 5, 6 $ cyclic group generated by 3. \\
    $F = $ Quaternions, $\bH = a+bi + cj + dk$ not a field, is a division ring \\
    $\bH $ contains a finite subset $Q_8  =\{ \pm 1, \pm i, \pm j, \pm k\}$ not cyclic \\
    $x^2+1=0$ has infinite roots $a_i+ b_j + c_k$ with $a^2+b^2+c^2=1$ \\
    $F = \bC$, polynomial $x^2-1=0$ has $\infty$ solutions
 \end{example}

 \noindent
 Useful Fact: If polynomail of $\deg \le n$ has $>n$ roots it is 0 \\
 Warning: A polynomial $f$ can vanish at all points of a field, still not be 0. \\
 $F$ = finite field, $\bZ/p\bZ$. \quad $f(x)=x^p-x$, roots all points of $F$ \\

 \noindent
 Polynomials over rationals form a UFD. \\
 What about integers. $\bZ[x]$ has no diivsion with remainder, not euclidian. Not all ideals principle. Consider $I =$ polynomials with even constant terms $(2 \, x)$ \\

 \noindent
 $6x^2 - 18x + 12 = 6(x^2-3x+2) = 2 \cdot 3 (x-2)(x-1)$. Note we have a factorization into irreducible polynomials such that coefficents have no common factors, primes of $\bZ$ \\
 We will show: 
 \begin{enumerate}
    \item Irreducible polynomials $\bZ[x]$ are prime 
    \item primes of $\bZ$ are prime in $\bZ[x]$ 
 \end{enumerate}
 We define the content $c(f)$ of a polynomia lin $\bZ[x]$ is the largest integer such that $\frac{f(x)}{c(x)}$ in $\bZ[x]$ = common divisor of all coefficents, eg. $c(6x^2-18x+12)=6$ \\
 Key property: $c(f)c(g)  =c(fg)$ \\ 
 Obvious: $c(f)c(g) \le c(fg)$ \quad Problem: $c(f)c(g) \ge c(fg)$ \\
 Divide $f, g$ by $c(f), c(g)$ to get polynomial with $c(h)=1$. Need to show that if $c(f)=1, c(g)=1$ then $cf(g)=1$ \\
 Suppose $p |c(fg)$, ($p$ prime) $p \not| c(f)$, $p \not| c(g)$. $f = a^nx^n + \cdots + a_ix^i + a_{i-1}x^{i-1} + \cdots a_0$, $g = b_mx^m + \cdots + b_jx^j = b_{j-1}x^{j-1} + cdots + b_0$ with $a_{i-1}, \ldots, a_0, b_{j-1}, \ldots, b_0$ divisible by $p$, $a_i, b_j$ not divsible by $p$. Now look at $fg$, the coefficent of $x^{i+j}$, $a_{i+j}b_0 + a_{i+j-1}b_1 + \cdots + a_ib_j + a_{i-1}b_{j+1} + \cdots + a_0b_{i+j}$. All terms except $a_ib_j$ divisble by $p$ so the coefficient of $x^{i+j}$ is not divisble by $p$. So prime does not divide $c(fg)$ so $c(fg)=1$, since $p$ was any prime. \\

 \noindent
 We can show that $\bZ[x]$ has unique factorization. Follows from: 
 \begin{enumerate}
    \item $\bQ[x]$ has unique factorization 
    \item $c(fg)=c(f)c(g)$ 
 \end{enumerate}
Key Steps: Show that if $f$ is a prime of $\bZ$ or polynomial of $\bZ[x]$ irreducible in $\bQ[x]$ with $c(f)=1$, then $f$ is prime, ie. if $f$ divides $gh$, $f$ divides $g$ or $f$ divides $h$. \\
2 cases: 
\begin{enumerate}
    \item $f$ prime of $\bZ \to$ if $p|gh$, $p|c(g)c(h)$ so $p|c(g)$ or $p|c(h)$ so $p|g$ or $p|h$ 
    \item $f$=poly, $c(f)=1$ similar 
\end{enumerate}

\noindent
Bonus: If ring $R$ has unique factorization so does the ring of polynomials $R[x]$ 

\begin{pf}
    Same proof but with $R = \bZ$. Key point: define content $c$, $c(fg) = c(f)c(g)$ 
\end{pf}    

\noindent
Can extend this: $k[x,y]$ polynomial in 2 variables. $k[x,y] = k[x][y]$ with $k[x]$ UFD so $k[x][y]$ is a UFD. \\
Repeating this: $k[x_1, \ldots, x_n]$ is a UFD. Still holds for polynomials in infinite variables as each polynomial only contains finitely many variables. 

\noindent
Problem: Given polynomial in $\bQ[x]$ or $\bZ[x]$ 
\begin{enumerate}
    \item Is it irreducible? 
    \item Factor into irreducibles 
\end{enumerate}
Is there an algorithn for this? yes - kronecker's algorithm \\
Recall: If a polynomial has $> \deg f$ roots, it is 0. If $f,g$, $\deg \le n$ and same at $>n$ points, they are equal. If $f=gh$, 

Bad news: This is really slow (not polynomial time) \\
Problem 1: Need to factor integers 
Problem 2: High number of possibilities for $g$ \\
Laadf adfadf found polynomial time algorithm for $\bQ[x]$ \\
can extend algorithm to $\bZ[x_1, \ldots, x_n]$ \\
Similar problem: Does polynomial $f(x_1, \ldots, x_n)$ in $\bZ[x_1, \ldots, x_n]$ have roots? No algorithm 

\noindent
Easy Checks for small polynomials in $\bZ[x]$ 
$f(x)$ is irreducible if leading coefficient is 1, irreducible modulo $p$, $f = gh \to f=gh \mod p$ 

\begin{example}
    $x^4-3x^3 + 2x-5$ is irreducible since irreducible modulo 2, $(x^4 + x^2+1)$ 
\end{example}

\noindent
Warning: Some polynomials look irreducible but are reducible (Auslaflllll polynomials) \\
$x^4 + 4y^4 = (x^2+2xy+2y^2)(x^2 - 2xy + 2y^2)$, however $x^4+y^4$ irreducible \\
Landry: factored 