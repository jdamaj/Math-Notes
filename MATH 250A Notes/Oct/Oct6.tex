
\subsection{Representations of Finite Abelian Groups}

We make the following observations about the character table of $S_3$ 
\begin{enumerate}
    \item Columns are orthogonal (under $\sum_{\chi} \chi(g)\overline{\chi(h)}=0$ $g,h$ not conjugate, $|G|$, $g,h$ conjugate) 
    \item \# columns = \# rows (\# conjugacy classes = \# irreducible reps) 
    \item Rows are orthogonal (under $\sum_{g}\chi_i(g)\overline{\chi_j(g)} = 0$, $i \neq j$, =$|G|$, $i=j$ = $\sum_{\text{conj classes } \{g\}} \chi_i(g)\overline{\chi_j(g)})\times $(size of conjugacy class)
\end{enumerate}

\noindent
Problem: Given a finite abelian group find the character table \\
Observation: All irreducible representations are one dimension \\
Reason: Pick some $g \in G$. $g$ acts on $V$ has an eigenvector with eigenvalue $\lambda$. Look at $V_{\lambda}$=all vectors with eigenvalue $\lambda$. $V_{\lambda}$ acted on by $G$. If $h \in G$, $v_{\lambda} \in V_{\lambda}$, $hv_{\lambda} \in V_{\lambda}$ since $g(hv_{\lambda}) = h(g v_{\lambda}) = \lambda h v_{\lambda}$ so $V=V_{\lambda}$ as $V$ is irreducible. \\

\noindent
So linear representations of $G$ are ``same as'' homomorphisms $G \to \bC^*$. $1 \in G \to $ some $z$ with $ z^n=1$, $n$th root of unity. \\
Dual group of $\bZ/n\bZ$ is isomorphic to $\bZ/4\bZ$ $a \mapsto e^{2 \pi i a b/n}$ $b=0, 1, \ldots, n-1$ \\
If $G$ is cylic, $G \cong \hat{G}$ but no natural isomorphism since depends on choice of generator and root of 1. \\
Any finite abelian groups is is a product of cylic groups $G  =\bZ/n_1\bZ \times \bZ/n_2\bZ \times \cdots$ \\
$\Hom(G, \bC^*) \ifff \Hom(\bZ/n_1\bZ, \bC^*) \times \Hom(\bZ/n_2\bZ, \bC^*) \times \cdots$ so $\hat{G} \cong \hat{\bZ/n_1\bZ} \times \hat{\bZ/n_2\bZ} \times \cdots \cong \bZ/n_1\bZ \times \bZ/n_2\bZ \times \cdots \cong G$. Any finite  abelian group is isomorphic to its dual (not canonically).

\noindent
Typical character tables: \\
$\bZ/5\bZ$: 
\begin{tabular}{ | c c c c c|}
    1 & 5 & 5 & 5 & 5 \\ 
    1 & 1 & 1 & 1 & 1 \\ \hline 
    1 & $z$ & $z^2$ & $z^3$ & $z^4$ \\ 
    1 & $z^2$ & $z^4$ & $z$ & $z^3$ \\ 
    1 & $z^3$ & $z$ & $z^4$ & $z^2$ \\
    1 & $z^4$ & $z^3$ & $z^2$ & $z$ \\  \hline 
\end{tabular} \quad
$\bZ/2\bZ \times \bZ/2\bZ$: 
\begin{tabular}{|c c c c|}
    1 & 2 & 2 & 2 \\ 
    1 & $a$ & $b$ & $c$ \\ \hline     
    1 & 1 & 1 & 1 \\ 
    1 & 1 & -1 & -1 \\ 
    1 & -1 & 1 & -1 \\ 
    1 & -1 & -1 & 1 
\end{tabular} \\

\noindent
Vector spaces: $V \cong V^*$ not canonical, $V \cong V^**$ canonical, $v \in V, v^* \in V^*$ $v(v^*) = v^*(v)$ \\
$G$ finite abelian, $G \cong \hat{G}$ not canonical, $G \cong \hat{\hat{G}}$ canonical: $g \in G, \hat{g} \in \hat{G}$ homomorphism by $\hat{G} \to G$ by $g(\hat{g}) \to \hat{g}(g)$ 

\noindent
Check Properties of character tables: 
\begin{enumerate}
    \item Table is square: \# conjugacy classes = \# irreducible representations since $|G| = |\hat{G}|$ 
    \item Rows orthogonal: want to show $\sum_{g} \chi_i(g)\overline{\chi_j(g)} = \begin{cases} |G| \quad i=j \\ 0 \quad i \neg j \end{cases}$. $\overline{\chi_j(g)} = \chi_j(g)^{-1}$ since $|\chi_j(g)|=1$ so suffices to show $\sum_{g} \chi(g) = \begin{cases} G \quad \chi \text{ trivial} \\ 0 \quad \chi \text{ nontrivial} \end{cases}$. Pick some $h$ with $\chi(g) \neq 1$. $\sum_{g} \chi(hg) = \sum_{g} \chi(h)\chi(g) = \chi(h)\sum_{g}\chi(g)$ and $\sum_{g}\chi(hg) = \sum_{g}\chi(g)$ so $(1 - \chi(h))\sum \chi(g)=0$ so since $\chi(h) \neq 1$, $\sum_g \chi(g)=0$ 
    \item Columns orthongonal 
\end{enumerate}

\noindent
So characters of $G$ from an orthogonal basis for the vector space of all complex functions on $G$. So for function $f$ from $G$ to $\bC$ we have $\sum a_{\chi} \chi(g)$, $a_{\chi} = (f, \chi) = \sum f(g) \overline{\chi{g}}$. $a_{\chi}$ called fourier coefficients. \\

\noindent
Fourier analysis: $f$ periodic, $f(x + 2 \pi) = f(x)$. $f = \sum_{n >0}a_n \sin (nx) + \sum_{n \ge 0 }b_n \cos (nx)$. $G$= group, $R/2\pi \bZ$. Dual group of $G$= homomorphisms from $G$ to $\bC^n$. $\hat{C} = \bZ$ by $x \mapsto e^{inx}$ $(n \in \bZ)$ \\
$e^{i n x} = \cos n x + i \sin n x = \sum c_n e^{inx}$, $\hat{\hat{G}} = G$. $\Hom(\bZ$, complex numbers with $|z|=1$) \\
$G = \bR, \hat{G} = \Hom(\bR, S^1)$ $x \mapsto e^{i \pi x y}$ $y \in \bR$ so $\hat{G} \cong G$ \\
Fourier Transform: $\int y \hat{f}(y)e^{i \pi xy}dx$  \\
Specific cases of pantiyagin duality: $G$ = locally compact abelian group, $\hat{G}$ = maps to $S^1$, $G \cong \hat{G}$ 

\noindent
What happens if field is not $\bC$? 
\begin{enumerate}
    \item Field has characterstic 0 but is not algebraically closed. Can get irreducible representations of dim >1 \\ 
    Ex: Field $F = \bR$, $G = \bZ/3\bZ$ \\
    Over $\bC$: 
    \begin{tabular}{| c c c |} 
        \hline \\ 
        1 & 1 & 1 \\ 
        1 & $\omega$ & $\omega^2$ \\ 
        1 & $\omega^2$ & $\omega$ \\ \hline     
    \end{tabular} \quad \\
    Over $\bR$: 
    \begin{tabular}{| c c c|}
        1 & 1 & 1 \\ 
        2 & -1 & -1
    \end{tabular}
    \item Char >0. Syppose characteristic is $p>0$. Look at maps of $\bZ/p\bZ$. Only irreducible representation is trivial one. Only possible eigenvalues is $\lambda = 1$ since $\lambda^p=1, (\lambda^p-1) = (\lambda-1)^p$ \\
    Look at representations that are indecomposable but not irreducible. Decomposable $\ifff$ linear transformation $T^p=1$ ie. $(T-1)^p=0 \ifff$ nilpotent matrices with $N^p=0$. $(0), \begin{pmatrix} 0 & 1 & \\ & 0 & 1 \\ & & 0 \end{pmatrix}, \ldots, \begin{pmatrix} 0 & 1 & & \\ 0 & \ddots & \ddots & \\ & & \ddots 1 & \\ & & & 0 \end{pmatrix}$ so we have $p$ distinct representations. 
\end{enumerate}

\noindent
Application: Dirichlet's Theorem: Given arithmetic progression $an+b$, $(a,b)=1$ contains $\infty$ primes. \\
Ex: $\infty$ primes of the form $4n+1$. We consider the character table of $(\bZ/a\bZ)^*$ \\
$a = 4$: \begin{tabular}{| c c c |} & 1 & 3 \\ \hline  $\chi_0$ & 1 & 1 \\ $\chi_1$ & 1 & -1  \end{tabular} \\
Dirichlet $L$-series: $\sum_{n} \frac{\chi(n)}{n}: \frac{1}{1^s} + \frac{1}{3^s} + \frac{1}{5^s} + \frac{1}{7^s} + \cdots$, $\frac{1}{1^s} - \frac{1}{3^s} + \frac{1}{5^s} - \frac{1}{7^s} + \cdots$ \\ 
$\sum_{n} \frac{\chi(n)}{n} = \prod_p \frac{1}{1 - \chi(p)p^s}$ so $\log \left( \sum \frac{\chi(n)}{n} \right) = \sum_{n.p} \frac{\chi(p^n)}{p^{ns}n}$ so we get  $ \frac{1}{3^s} + \frac{1}{5^s} + \frac{1}{7^s} + \frac{1}{2 \cdot 9^s}+ \cdots$, infinite at $s=1$, $ - \frac{1}{3^s} + \frac{1}{5^s} - \frac{1}{7^s} + \frac{1}{2 \cdot 9^s} + \cdots$ finite at $s=1$, nonzero since series converges to $\frac{\pi}{4} \neq 0$ \\ 
Define $f=1$ if $n \equiv 1 \mod 4$, 0 otherwise. Function on $(\bZ/4\bZ)^*$ = linear combinations of $\chi_0, \chi_1$ = $\frac{1}{2}(\chi_0 + \chi_1)$. $\frac{1}{2}$ sum if $\frac{1}{5^s} + \frac{1}{2 \cdot 9^s} + \frac{1}{13^2} + \cdots = \sum_{n,p \equiv 1 \mod 4} \frac{1}{p^{ns}n}$ is infinite on $s=1$. Sum of terms $\frac{1}{np^{ns}}$ $n \ge 2$ is finite so $\sum_{p \equiv 1 \mod 4} \frac{1}{p} = \infty$ \\
Key point: $\sum \frac{\chi(n)}{n^s} \neq 0$ at $s=1$ (if $\chi \neq $ trivial) \quad hard step 
