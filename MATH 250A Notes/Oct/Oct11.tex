
\subsection{Orthogonality relations} 

Character Tables: 
\begin{itemize}
    \item rows orthogonal: weight by size of conjugacy classes
    \item Norm of row = $|G|$ 
    \item columns orthogonal 
    \item norm of columns = $|G|$/(size of conjugacy class)
\end{itemize}

\noindent
Special Cases 
\begin{enumerate}
    \item \# conjugacy classes = \# characters 
    \item $\sum d_i^2 |G|$ ($d_i$ = dimension of irreducible characters) 
\end{enumerate}

\noindent
\textbf{Quaternion Group} 
\begin{itemize}
    \item Find 1-dimensional elements $\equiv$ same as characters of abelianized group = $G$/(normal subgroup generated by $ghg^{-1}h^{-1})$ \\
    Note that this is adjoint to the forgetful functor. 
    \item Abelianization of $Q = Q/\{\pm 1\} = (\bZ/2\bZ)^2$ - 4 characters 
    \item Use orthogonality relations, $\sum d_i^2 = G$. $1^2 + 1^2 + 1^2 + 1^2 + d^2=8$, $d=2$. Last rep given by row orthogonality. 
\end{itemize}

\[\begin{tabular}{| c c c c c |}
    1 & -1 & $\pm i$ & $\pm j$ & $\pm k$ \\ 
    1 & 1 & 2 & 2 & 2 \\ \hline 
    1 & 1 & 1 & 1 & 1  \\ 
    1 & 1 & -1 & -1 & 1 \\ 
    1 & 1 & -1 & 1 & -1 \\ 
    2 & -2 & 0 & 0 & 0 \\ \hline
\end{tabular} \]

\noindent
Dihedral group of order 8 has same character table as $Q$. Possible for different groups to share the same character table. 

\noindent
\textbf{Alternating Group} - $A_4$ 
\begin{itemize}
    \item Use permutation represntation of $A_4$ of 4 points so 4 dimension representation with this basis. 
    \item What is its characeter? What is the character of permutation representation (on $n$ points) of an $n$ dimension vector space? \\
    trace = \# fixed points so permutation rep has character $(4,0,1,1)$, not irreducible. \\
    \item How many copies of 1-dimensional element. $( \, , \,)=12=|G|$ with trivial character so can subtract out to get $(3,-1,0,0)$, norm = 12 - so is irreducible. 
\end{itemize}

\noindent
\textbf{$S_4$} 
\begin{itemize}
    \item Abelianization  = $\bZ/2\bZ$ 
    \item Permutation representation: $(4, 0, 2, 0, 1, 0)$ reduce to irreducible representation $(3, 1, -1, 0, 1)$ 
    \item Have product of 3 dimensional representation with 1 dimensional representation 
\end{itemize}

\[\begin{tabular}{|c c c c c|}
    1 & (1\, 2) & (1 \, 2)(3 \, 4) & (1 \, 2 \, 3) & (1 \, 2 \, 3 \, 4) \\
    1 & 6 & 3 & 8 & 6 \\ \hline 
    1 & 1 & 1 & 1 & 1 \\
    1 & -1 & 1 & 1 & -1 \\ 
    3 & 1 & -1 & 0 & 1 \\ 
    3 & -1 & -1 & 0 & 1 \\
    2 & 0 & 2 & -1 & 0 
\end{tabular}\]

\noindent
Abelian groups: If $\chi_1, \chi_2$ are irreducible characters so is $\chi_1 \chi_2$ \\
Non abelian group: $\chi_1, \chi_2$ usually not irreducible. It is irreducible if $\chi_1$ has dimension 1. \\
If $G$ acts on $V$, $\chi_1$ a character, we get a representatio no $V$ by $g \mapsto \chi_1(g) \cdot g$ \\

\noindent
Finding Normal subgroups from character tables \\ 
Suppose $V$ is an irreducible representaion of dimension $d$, character $\chi$. What is $\chi(g)?$ Diagonalize $g$, diagonal entries = roots of 1. $\chi(g) = z_1 + z_2 + \cdots + z_d$ where $z_i$ is a root of 1. Now, $|z_1 + \cdots + z_n|\le d$, equality holds if all $z_i$ are equal. if $z_1 + \cdots + z_n = d$, all $z_i=1$ so if $\chi(g) = \chi(1)$, $g$ acts trivially on rep. \\
For $S_4$, element $(1), (1 \, 2)(3 \, 4) +$ conjugates act trivially in 2 dimensonal representation, form a normal subgroup. 

\begin{example}
    Binary Dihedral group of order 24 $\stackrel{\text{onto}}{\to} A_4$ so get representations of dimension $1,1,1,3$ 
\end{example}

\begin{example}
    $A_5$ = alternating group = rotations of icosahedron\\
    $A_5$ acts on $\bR^3$ so get 3-dimensional representations with characters as trace of rotations of icosahedron \\
    Use outer automorphism $A_5 \subseteq S_5$ to get a rep \\ 
    Now, $1^2 + 3^2 + 3^2 + x^2+y^2 = 60$ so $x=4,y=5$. Perm rep $(5, 1, 2, 0, 0)$ with irreducible $(4, 0, 1, -1, -1)$. 
    \[\begin{tabular}{|c c c c c|}
        1 & (1 \,2)(3 \, 4) & ( 1 \, 2 \, 3) & (1 \, 2 \, 3 \, 4 \, 5) & (1 \, 2 \, 3 \, 5 \,4)\\ \hline 
        1 & 1 & 1 & 1 &  1\\ 
        3 & -1 & 0 & $1 - 2\cos \frac{2 \pi}{5}$ & $1 - 2\cos \frac{4 \pi}{5}$ \\ 
        3 & -1 & 0 & $1 - 2\cos \frac{4 \pi }{5}$ & $1 - 2 \cos \frac{2 \pi }{5}$ \\ 
        4 & 0 & 1 & -1 & -1 \\ 
        5 & 1 & -1 & 0 & -0\\ \hline 
    \end{tabular} \]
\end{example}

\begin{example}
    $S_5$, binary dihedral group of order 120, symmetry group $S_6$ 
\end{example}

\subsection{Proofs Of Orthogonality Relations}

\begin{enumerate}
    \item All representations of $G$ can be made unitary $( \, , \, )$ invariant under $G$. \\
    Define $( \,, \,)$ by taking any $( \, , \,)$ make invariant under $G$ be taking average 
    \item Want to show if $\chi$ is an irreducible character, $(\chi, 1 ) = 0$, $\sum_g \chi(g)=0$ \\
\end{enumerate}

\noindent
Suppose $V$ is an irreducible representation (finite dimensional) with no fixed vectors $(\neg 0)$, then $\sum_g \chi(g)=0$. This holds for all irreducible representations except fo the trivial one. Pick $v \in V$, $\sum_{g \in G}g(v)=0$ since no fixed vectors $\neq 0$ 

\begin{enumerate}
    \setcounter{enumi}{2}
    \item Suppose $V, W$ are irreducible representaions. Look at vector space $\Hom(V, W)$. Have rep by $G$, $\dim = \dim V \times \dim W$ character $\chi_W \overline{\chi_V}$, $\chi_V, \chi_W$ characters of $V,W$. Enough to check for one factor $\chi_{\Hom(V,W)}(g) = \chi_{W}(g) \chi_V(g)$. Choose basesof $V,W$ such that $g$ is diagonal. Split $V,W$ into sum of 1-dimensional spaces acted on by $g$. Suffices to show case where $V,W, \dim = 1$ 
\end{enumerate}

\noindent
Schur's Lemma: Suppose $V,W$ irreducible representations, then $\Hom_G(V,W)$, homomorphisms invariant under $g$ ie. fixed points of $G$ on $\Hom(V,W)$ has dimension $\begin{cases} 1 \quad \text{ if } V \cong W \\ 0 \quad \text{ if } V \not\cong W \end{cases}$. \\
$V \to W$ invariant under $G$, image is invariant subspace so is 0 or $W$ as $W$ is irreducible. Kernerl is invariant subspace of $V$ so is 0 or $V$ as $V$ is irreducible. so map is either 0 or isomorphism. \\
If $V,W$ not isomorphic, no maps $V \to W$ invariant under $G$. \\
If $V = W$, then $\Hom(V,V)$ is a division algebra (eg. ring where elements have inverse), finite dimensional algebra over $\bC$, algebraically closed, any division algebra is $\bC$. So $\Hom_G(V,W) = \bC$ 

\begin{example}
    Look at real reps of $\bZ/3\bZ$ \\
    \begin{tabular}{| c c c |} 1 & $g$ & $g^2$ \\ 1 & 1 & 1 \\ 1 & $\omega$ & $\omega^2$ \\ 1 & $\omega^2$ & $\omega$ \end{tabular} \quad \begin{tabular}{| c c c |} 1 & $g$ & $g^2$ \\ 1 & 1 & 1 \\ 2 & -1 & -1 \end{tabular} \\
    $\Hom(V,V) = \bC$, $\bC$ division algebra over $\bC$ 
\end{example}

\begin{example}
    $G = Q_8$ acts on quaternions $H$ by left multiplication. 4-dim real representation, 2-dim complex representation. $\Hom_G(V,V)=H$ action given by right multiplication. $H$ division algebra over $\bR$. 
\end{example}

\noindent
Row orthogonality: If $V,W$ irreducible, then $\sum_g \chi_V(g) \overline{\chi_W(g)}=0$ $V \not\cong W$ $|G|$ if $V \cong W$. Look at character $\Hom(V,W) = \chi_V \overline{\chi_W}$. If $W \not\cong V$, $\Hom(V,W)$ doesnt contain any invariant characters so is 0. So $\sum_g \chi_v(g)\overline{\chi_w(g)}=0$. If $V = W$, $\Hom(V,V)$ is a 1-dimensional subspace so $\sum_g \chi_V(g)\overline{\chi_W(g)}=|G|$ \\

\begin{corollary}
    Any representation is determined by its characters. 
\end{corollary}

$V = \bigoplus V_i$ ($V_i$ irreducible) by complete irreduciblility. By orthogonality, number of irreducible representatiosn $W$ appears is $\frac{(\chi_W, \chi_V)}{|G|}$. \\
Fails over field with char $>0$ \\ 
$G = \bZ/p\bZ$ Field = $\bZ/p\bZ$. Rep: $g \to \begin{pmatrix} 1 & 0 & 0 & 1 \end{pmatrix}$ trivial 2-dim representation, $g^n \to \begin{pmatrix} 1 & n & 0 & 1 \end{pmatrix}$ indecomposable. 