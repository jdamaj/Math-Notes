
\subsection{Langrange's Theorem}

\textbf{Order 4}: $\bZ/4\bZ$, symmetries of rectangle \\
How to show not isomorphic? \\
Find some property (preserved by isomorphism) that one group has but the other does not. \\

\noindent
Property: Order of elements 
\begin{itemize}
    \item in $\bZ/4\bZ$, 0, 1, 2, 3 have orders 1, 4, 2, 4 respectively 
    \item all nontrivial elements of the group of symmetries of the rectangle have order 2
\end{itemize}
Note: counting elements of each order works for small gorups but 2 groups of order 16 with same number of elements of each order \\

\noindent
Classification: By Lagrange's theorem, each element has order 1, 2, or 4 
\begin{enumerate}
    \item Have an element of order 4: $g$, group = $\{1, g, g^2, g^3\}\cong \bZ/4\bZ$ \\
    In general, if a group of $n$ elements has an element of order $n$, it is $\cong \bZ/4\bZ$
    \item All elements have order 1 or 2. \\
    Suppose $G$ is finite and has this property. Then $G$ commutes since $(gh)^2 = ghgh=1=g^2g^2$ so $gh = hg$. \\
    Note: only true for prime 2, there is a group of order 27 such that all elements have order 1 or 3 but is not commutative \\
    Write group operation as $+$. $G$ is a vector space over $\bF_2$ (field of 2 elements). So $G \cong \bF_2^k$ for osme set $|G|=2^k$. We get 1 group of order 4 with all elements of order 1 or 2. 
\end{enumerate}

\noindent
Group of order 4 is product of 2 groups, $\bF_2^2 = \bF_2 \oplus \bF_2$. \\
Suppose $G, H$ are gorups, $G \times H$ is a gorup under operation $(g_1, h_1) \cdot (g_2, h_2) = (g_1g_2, h_1h_2)$

\begin{example}
    $\bC^{\times} \cong \bR_{\ge 0} \times S^1$, $z = |z| \cdot e^{i \theta}$
\end{example}

\noindent
Chinese Remainder Theorem: $(m, n)$ coprime, $\bZ/mn\bZ \cong \bZ/m\bZ \times \bZ/n\bZ$. \\
We have maps $f: \bZ/mn\bZ \to \bZ/m\bZ$, $g : \bZ / mn\bZ \to \bZ/n\bZ$. This gives $h : \bZ/mn\bZ \to \bZ/m\bZ \times \bZ/n\bZ$. If $(m, n)=1$, then the map is injecitve since if $h(k)=0$, $k \equiv 0 \mod m, \mod n$ \\

\noindent
Infinite Products: $G_1 \times G_2 \times G_3 \times \cdots $,\quad set of all elements $(g_1, g_2, g_3, \ldots, )$ \\
Infinite Sums: Like infinite products but all but finitely many of $g_1$ are 1. 

\begin{example}
    Roots of 1 = $e^{2 \pi q}$, $q \in \bQ$. \\
    Infinite sum $G_2 + G_3 + G_5 + G_7 + G_11 + \cdots$ \quad $(G_p$ = roots of order $p^n$ for some $n \ge 1$)
\end{example}

\noindent
Symmetry of Platonic Solids \\
\begin{tabular}{c c c l}
    Faces & Name & Rotations & Rotations + Reflections  \\
    4 & tetrahedron & $12 = 4 \times 3$ & \quad 24 \quad $\to$ not a product \\
    $\begin{array}{c}
       6 \\ 8 \\ 12 \\ 20
    \end{array}$ & 
    $\begin{array}{c}
       \text{hexahedron (cube)} \\ \text{octahedron} \\ \text{dodecahedron} \\ \text{icosahedron}
    \end{array}$ & 
    $\begin{array}{c}
       24 = 6 \times 4 \\ 24 = 8 \times 3 \\ 60 = 12 \times 5 \\ 60 = 20 \times 3
    \end{array}$ & 
    $\left. \begin{array}{c}
       48 \\ 48 \\ 120 \\ 120
    \end{array} \right\} \text{product } \bZ/2\bZ \times \text{rotations} $
\end{tabular}
All except tetrahedron have symmetry $\begin{pmatrix} -1 & & \\ & -1 & \\ & & -1 \end{pmatrix}$ fo reflections in $\bR^3$, so it commutes with everything \\
For the tetrahedron, we have $\begin{pmatrix} -1 & & \\ & 1 & \\ & & 1 \end{pmatrix}$ \\

\noindent
\textbf{Order 5}: $\bZ/5\bZ$
\begin{exercise}
    Find a graph as small as possible with symmetries $\bZ/5\bZ$ 
\end{exercise}

\noindent
\textbf{Order 6}: 3 obvious examples: $\bZ/6\bZ$, $\bZ/2\bZ \times \bZ/3\bZ$, symmetries of the triangle \\
\begin{itemize}
    \item $\bZ/6\bZ \cong \bZ/2\bZ \times \bZ/3\bZ$ 
    \item group of symmetries of the triange is not abelian \\
    Permutation Notation: $(5 \, 2 \, 1 \, 3)=$ function sending $5 \to 2$, $2 \to 1$, $1 \to 3$, $3 \to 5$\\
    (Insert Figure) \\
    $(1 \, 2) (2 \, 3) = (1 \, 2 \, 3)$ but $(2 \, 3)(1 \, 2) = (1 \, 3 \, 2)$ 
\end{itemize}

\begin{definition}
    A subgroup of a group $G$, is a subset closed under group operations.
\end{definition}

\begin{theorem}[Lagrange's Theorem]
    If $H$ is a subgroup of $G$, $|H|$ divides $|G|$.
\end{theorem}
Special Case: If $H=$ powers of $g$, $1, g, g^2, \ldots, g^{n-1}$, $|H|=|g|$ \\

\noindent
Construction of subgorups: Pick a set $S$ acted on by $G$, pick $s \in S$. \\
$H$: elements $g$ with $gs=s$ (elements fixing $s$). Then $H$ is a subgroup. \\
Lagrange (Converse to Cayley's Thm): If $H$ is a subgroup of $G$ we can find a set acted on by $G$, such that $H$=elements fixing $s \in S$. \\

\noindent
Given a gorup $G$, subgroup $H$. We want to construct: a set $S$ acted on by $G$. \\
Consider $G$=symmetries of triangle, $H = \{ (1)(2)(3), (2 \, 3)\}$ fixing 1. \\
How do we write 1, 2, 3 in terms of $G, H$? \\
Left cosets of $H$: $1 \ifff$ elements $g$ with $g(1)=1$ (H), $2 \ifff $ elements $g$ with $g(1)=2$ ($(1 \, 2)H$), $3 \ifff $ elements $g$ with $g(1)=3$ ($(1 \, 3)H$) \\

\noindent
Left cosets of $H$ are sets of the from $aH$ (some fixed $a \in G$). \\
Define $g_1 \approx g_2$ if $g_1 = g_2h$ for some $h \in H$. This is an equivalence relation: \\
Reflexivity: $g_1 \approx g_1$ \quad group identity, 1 \\
Symmetry: $g_1 \approx g_2 \to g_2 \approx g_1$ \quad group inverses, $h^{-1}$ \\
Transitivity: $g_1 \approx g_2, g_2 \approx g_3 \to g_1 \approx g_3$ \quad group operation, $h_1h_2$ \\
$G$ = disjoint union of cosets (equivalence classes of $\approx$) and any two cosets have the same same $|H|$ since we have a bijection $H \to aH$ byb $h \mapsto ah$ with inverse $h \mapsto a^{-1}h$. \\
So $G =$ \# cosets $\times$ size of cosets = $\#$ elements of $S$ $\times$ |subgroup of elements fixing $s$| \\
Note: We assume $S$ is transisitve - if $s_1, s_2 \in S$. $g(s_1)=s_2$ for some $g$ \\

\noindent
Rotations of a dodecahedron: 12 (faces) $\times$ 5  = 20 (vertices) $\times$ 3 = 30 (edges) $\times$ 2 = 60 \\

\noindent 
Conways Group: has order 831555361308172000 \\
Acting on Frames: \# 8252375 \quad Group fixing each frame: 1002795171840 \\

\noindent
Special Cases of Lagrange: 
\begin{itemize}
    \item Fermat: $a^p \equiv a \mod p$ ($p$ prime), $a^{p-1} \equiv 1 \mod p$  $(a,p)=1$ \\
    Group $(\bZ/p\bZ)^{\times}$ integers modulo $p$ under $\times$ has order $p-1$. \\
    Lagrange: order of $a$ divides $p-1$ so $a^{p-1}\equiv 1$
    \item Euler: $a^{\vp(m)} \equiv 1 \mod n$ $(a,m)=1$ \\
    $(\bZ/m\bZ)^{\times}$= group of elements coprime to $m$, mod $m$, order = $\vp(m)$
\end{itemize}
$m=8$: $\vp(m)=4$, $(\bZ/8\bZ)^{\times} = \{1, 3, 5, 7\}$. Euler $a^4 \equiv 1 \mod 8$ ($a$ odd) but we see $a^2 \equiv 1 \mod 8$ \\

\noindent
Right Cosets: $Ha \ifff$ elements of a set acted on, on the right by $G$. $S \times G \to S$ \\
Are left cosets the same as right cosets? sometimes 

\begin{example}
    Take $G$ = symmetries of triangle. $H = \{1, (2 \, 3)\}$. Find the left, right costs of $H$ in $G$. \\
    Left: $H = \{1 (2 \, 3)\}, (3 \, 1)H = \{(3 \, 1), (3 \, 2 \, 1)\}, (1 \, 2)H = \{(1 \, 2), (1 \, 2 \, 3)\}$ \\
    Right:  $H = \{1 (2 \, 3)\}, (3 \, 1)H = \{(3 \, 1), (1 \, 2 \, 3)\}, (1 \, 2)H = \{(1 \, 2), (3 \, 2 \, 1)\}$ \\
    so left cosets $\neq$ right cosets
\end{example}

\begin{definition}
    Index of $H$ in $G$, $[G:H]$= \# cosets of $H$ in $G$.
\end{definition}

\noindent
Left or right cosets? $[G:H][H] = |G|$ when $G$ finite so \# left cosets = \# right cosets. \\
In gernal, right cosets $\to$ left cosets by $Ha \mapsto a^{-1}H$ so \# left cosets = \# right cosets

\subsection{Normal Subgroups}

$G/H$ = set of left coset of $G$. Is $G/H$ a group? \\
How to definte $(g_1H) \times (g_2H)$? $g_1g_2H$ \\
Problem: not well defined - suppose we have $g_1, g_2, g_1h_1, g_2h_2$. Want $g_1g_2H = g_1h_1g_2h_2H$ \\
Is $h_1g_2 = g_2 (h \in H)$? not in general \\
Want: $ghg^{-1} \in H$ for all $g \in G$. If this holds, then we can turn $G/H$ into a group. 

\begin{definition}
    If $H$ satisfies the above property, $H$ is called a normal subgroup of $G$.
\end{definition}

\begin{example}
    $G$ = symmetries of triangle. $H = \{(2 \, 3), 1\}$. Is $H$ normal? \\
    $(1 \, 2)(2 \, 3)(1 \, 2)^{-1} = (1 \, 3) \not\in H$ so $H$ is not normal
\end{example}

What about $H = \{1, (1 \, 2 \, 3), (1 \, 3 \, 2)\}$. Is $H$ normal? \\
$H$ has index 2 in $G$. $[G:H] = \frac{|G|}{|H|}=2$. We claim any subset of order 2 is normal. \\
There are only 2 left cosets: $H$, things not in $H$. Similarly for right cosets. So right cosets = left cosets. So $H$ is normal. 

\noindent 
\textbf{Classifying Groups of Order 6}
\begin{itemize}
    \item orders of elements 1, 2, 3, 6
    \item If element of order 6, group must be cyclic
    \item Want element of order 3
\end{itemize}
Lagrange: order of element divides order of group \\
Converse: If $n$ divides $|G|$, does $G$ have a subgroup of order $n$? \\
No: $\bZ/2\bZ \times \bZ/2\bZ$ has no element of order 4 \\
Yes: if $n$ is prime (Cauchy) \\
So $G$ has elements $a,b$ of order 2,3 and subset $(1, b, b^2)$ has order 2 so it is normal. 