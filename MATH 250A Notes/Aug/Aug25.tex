
\subsection{Groups}

Two ways to define groups
\begin{itemize}
    \item concrete: group = symmetries of an object $X$. Here a symmetry is a bijection $X \to X$ with inverse that preserves ``structure'' (topology, order, binary operation, $\ldots$)
\end{itemize}

\begin{example}
    The rectangle has 4 symmetries. \\
    The icossahedron has $20 \times 3$ symmetries since after fixing the first face there are 3 possible rotations. \\
    Vector space $\bR^k$: $n \times n$ matrices with $\det \neq 0$, denoted $GL_n(K)$
\end{example}

\begin{itemize}
    \item abstract definition: \\
    \begin{definition}
        A group is a set $G$ with a binary operation $G \times G \to G$ by $(a,b) \mapsto ab, a \times , a + b, \ldots$ with ''Inverse'' : $G \to G$ by $a \mapsto a^{-1}$ and ``Identity'': $1, 0, e, I, \ldots$ satsifying the axioms: \\
        $1x = x1 = x$ \quad $x(x^{-1}) = (x^{-1})x = 1$ \quad $(xy)z = x(yz)$ 
    \end{definition}
\end{itemize}

\noindent
We can go from the concrete definition to the abstract one: the binary operation is composition, the identity is the trivial symmetry, inverses given y ``undoing' a symmetry. \\

\noindent
Is an abstract group the symmetries of something? 

\begin{theorem}[Cayley's Theorem]
    Any abstract group is the group of symmetries of some mathematical object. 
\end{theorem}

\noindent
Recall group actions : 

\begin{definition}
    Given a group $G$, a set $S$, a (left) gtroup action is a map $G \times S \to S$ by $(g,s) \mapsto g(s), gs$ satisfying $g(h(s)) = gh(s)$, $1s=s$.
\end{definition}

\noindent 
To prove Cayley's theorem we need to find : 
\begin{enumerate}
    \item a set $S$ acted on by $G$
    \item structure on $S$ so that $G=$ all symmetries.
\end{enumerate}

\noindent
What is $S$? \quad Take $S = G$. \\

\noindent
Need to define the action of $G on G$. There are 8 natural ways to do this. \\
First 4, we defin4 $G \times S \to S$ by 
\begin{itemize}
    \item $g(s)=s$ \quad trivial action 
    \item $g(s) = gs$ \quad group product 
    \item Try $g(s)=sg$ \quad Fails since $G$ not necesarily commutative: $g(h(s)) = (sh)g \neq s(gh) = gh(s)$
    \item $g(s) = sg^{-1}$ \quad works since $g(h(s)) = g(sh^{-1}) = sh^{-1}g^{-1} = s(gh)^{-1} = gh(s)$
    \item $g(s) = gsg^{-1}$ \quad adjoint action 
\end{itemize}

\noindent 
The above group action is known as a left group action, We define a right group action in a similar way : \\
$S \times G \to S$ by$(s,g) \mapsti (s)g, s^g$ satisfying (sg)h = s(gh), $s1 = s$. \\

\noindent
We now define right group actions of $G$ on $G$: $S \times G \to G$ by 
\begin{itemize}
    \item $(s,g) \mapsto s$
    \item $(s,g) \mapsto sg$
    \item $(s,g) \mapsto g^{-1}s$
    \item $(s,g) \mapsto g^{-1}sg$
\end{itemize}