
\subsection{Complete Theories} 

Observation: Let $f$ be a function $:k \to k$. If $f$ is one to one then $f$ is onto, provided $k$ is finite. \\

\begin{theorem}
    Every injective polynomial map $\bC^n \to \bC^n$ is surjective. \\
    (A polynomial map consists of $n$ polynomials $p_1[x_1, \ldots, x_n], \ldots, p_n[x_1, \ldots, x_n] \in \bC[x]$ )
\end{theorem}

\begin{lemma}
    Let $phi$ be a senctence in the language $\{0, 1, + , \times \}$. TFAE 
    \begin{enumerate}
        \item $C \models \phi$ 
        \item $\phi$ is true in any algebraically closed field of characteristic 0. 
        \item $\phi$ is true in some algebraically closed field of characteristic 0. 
        \item There are arbitrarily large primes $p$ such that $\phi$ is true in some $F \models \text{ACF}_p$ 
        \item There is an $m \in \bN$ such that for all $p \ge n$ and all $F \models \text{ACF}_p$, $F \models \phi$ 
    \end{enumerate}
\end{lemma}

\begin{pf}
    (1), (2), (3) equivalent since $\text{ACF}_0$ is complete. (4) $\to$ (5) clear. \\
    $(2) \to (5)$ $\text{ACF}_0 \models \phi$. There is finite $\Delta \subseteq \text{ACF}_0$ such that $\Delta \models \phi$. If $p \ge n$ for an all $n$ such that $``\overbrace{1 + \cdot + 1} \neq 0''$ shows up in $\Delta$, then if $F \models \text{ACF}_p$, $F \models \Delta$ so $f \models \phi$ \\
    $(4) \to (3)$ If $(3)$ was false, $\text{ACF}_0 \models \neq \phi$ and for some $n$, all $p > n$, if $F \models \text{ACF}_p$ then $F \models \neg \phi$ so (4) is false. 
\end{pf}

\noindent
Claim: Every injective polynomial function $f: (\bF_p^{\text{alg}})^n \to (\bF_p^{\text{alg}})^n$ is onto \\
where $\bF_p^{\text{alg}}$ is the algebraic closure of $\bF_p: \bZ/p\bZ$. $\bF_p^{\text{alg}} = \bigcup_{n \in \bN}\bF_{p^n}$ where $\bF_{p^n}$ is the unique field of size $p^n$. \\

\noindent
For every polynomial $p(\overline{x}) \in F$ there is an atomic $t(\overline{x}, \overline{z})$ and parameters $\overline{c} \in F$ such that $p(\overline{x}) = t(\overline{x}, \overline{c})$ so $t_1(\overline{x}, \overline{c}), \ldots, t_n(\overline{x}, \overline{c})$ for $\overline{c}\in \bF_p^{\text{alg}}$, $\overline{x} = x_1, \ldots, x_n$ \\
Claim states $\fa \overline{z}( \fa \overline{x} \fa \overline{y} \bigwedge_{i=1}^n t_i(x_i, z) = t_i(y_1, z) \to \overline{x} = \overline{y}) \to (\fa \overline{w} \ex \overline{x} \bigwedge_{i=1}^n t_i(\overline{x}, z)=w_i)$ 

\begin{pf}[Pf of Claim]
    Take $\overline{b} \in (\bF_p^{\text{alg}})^n$, want to show $\overline{b}$ is in the range of $f$ \\
    Let $k$ be the finite subfield of $\bF_{p}^{\text{alg}}$ generated by $\overline{c}$ and $\overline{b}$. $\bF_p(\overline{c}, \overline{d})$ \\
    Restricting $f$ to $k^n$, we get a one to one function from $k^n$ to $k^n$ so $f\upharpoonright k^n$ is onto so $\overline{b}$ is in the range of $f$
\end{pf}

\subsection{Up and Down}

\begin{definition}
    A map $j: \cM \to \cN$ is an elementary embedding if for all formulas $\phi(\overline{x})$, all $m \in M$ 
    \[\cM \models \phi(\overline{m}) \ifff \cN \models \phi(j(\overline{m})) \]
\end{definition}

\begin{definition}
    If for $\cM \subseteq \cN$, $\cM$ is an elementary subset of $\cN$ if $i: M \hookrightarrow N$ is elementary $(\cM \preceq \cN)$ 
\end{definition}

\begin{itemize}
    \item $\bQ^{\text{alg}} \preceq \bC$, $(\bQ, 0, + ) \preceq (\bR, 0, +)$ 
\end{itemize}

\begin{definition}
    Given $\cM$, let $\cL_M = \cL \cup \{ c_m \, | \, m \in M\}$. $\cM$ can be made into an $\cL_m$-structure $\cM^*$ by letting $c_m^{\cM^*} = m$
\end{definition}

\begin{definition}
   $\dia(\cM)$ the atomic diagram of $\cM = \{ \phi \, | \, \phi \text{ atomic } \cL_M \text{ sentence such that } \cM \models \phi \} \cup \{ \neg \phi \, | \, \phi \text{ atomic } \cL_M \text{ sentence such that }\cM \models \neg \phi \}$\\ 
   This is equivalent to $\{ \phi \, | \, \phi \text{ is an } \cL\text{-formula } \cM \models \phi\}$ \\
   $\diae(\cM)$, the elementary diagram of $\cM$ is $\{ \phi \, | \, \phi \text{ is an } \cL \text{formula } \cM \models \phi\}$  
\end{definition}

\begin{lemma}
    \begin{enumerate}[label = (\roman*)]
        \item if $\cN^* \models \dia(\cM)$ then there is an $\cL$-embedding $\cM \to \cN$ (where $\cN$ the restriction of $\cN^*$ to $\cL$)
    \end{enumerate}
\end{lemma}

\begin{pf}
    Suppose $\cN^* \models \dia(\cM)$. If $\phi(\overline{x})$ is an $\cL$ formula and $\overline{c_m}$ new constants, we can give an embedding by $m \mapsto c_m^{\cM^*}$ \\ 
    $\cM \models \phi(\overline{m}) \ifff \cM^* \models \phi(\overline{c_m}) \ifff \cN^* \models \phi(\overline{c_m}) \ifff \cN \models \phi(\overline{m})$ 
\end{pf}

\begin{example}
    $\cM = (\bZ, +)$, $\cL = \{*\}$, $\cL_M = \{*, c_0, c_1, c_2, \ldots, c_{-1}, c_{-2}, \ldots\}$, in $\cM^*$, $c_n^{\cM^*} = n$\\
    $\cN = (\bR, \times)$, define $\cN^*$ by $c_n^{\cN^*} = 2^n$. $\cN^* = (\bR, \times, c_n \mapsto 2^n)$ \\
    $\cN^* \models \dia(\cM)$ size $(\bZ, +) \to (\bR, \times)$ by $n \mapsto 2^n$ is an embedding. 
\end{example}

\noindent
If $j: \cM \to \cN$ is an embedding, let $c_m^{\cM^*} = j(m)$. Then $\cN^* \models \dia (\cM)$ 
